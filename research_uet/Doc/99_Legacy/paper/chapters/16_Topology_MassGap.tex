\chapter{Topology and the Mass Gap}
\label{ch:topology_massgap}

\section{The Yang-Mills Mass Gap (Topic 0.21)}
\begin{itemize}
    \item \textbf{Problem:} Proving that non-abelian gauge theories have a mass gap (particle mass > 0).
    \item \textbf{UET Solution:} \texttt{Engine\_Mass\_Gap.py} demonstrates that the Mass Gap is the \textbf{Minimum Information Packet Size}.
\end{itemize}

\section{Why Gluons Cannot be Massless}
In a discrete information grid, a massless field implies infinite range and infinite information capacity, which is physically impossible. The field must "knot" into localized solitons (Glueballs) to store information efficiently.
\begin{equation}
    \Delta \ge \hbar \omega_{min}
\end{equation}
The gap $\Delta$ is the energy cost to create the smallest possible topological knot in the field.
