\chapter{Black Holes and Singularities}
\label{ch:blackholes}

\section{The Singularity Problem}
General Relativity predicts its own demise at the center of a black hole, where curvature becomes infinite. UET resolves this by imposing a fundamental limit on the density of information storage in the manifold.

\section{The Supersonic Sink Model}
\begin{itemize}
    \item \textbf{Engine:} \texttt{Engine\_BlackHole.py} (Topic 0.2)
    \item \textbf{Core Concept:} A black hole is not a singularity, but a region where the flow of the information medium exceeds the speed of light ($v_{flow} > c$).
    \item \textbf{Resolution:} At the center, the Information Density ($I$) saturates the grid capacity ($\rho_{max}$). The manifold becomes incompressible, preventing infinite curvature.
\end{itemize}

\section{Thermodynamics and Hawking Radiation}
UET derives Hawking Radiation as an "Entropy Leakage" mechanism. As the event horizon scrambles information, the resulting entropy gradient ($\nabla S$) drives a flux of energy back into the universe, preserving unitarity.

\begin{figure}[h]
    \centering
    % \includegraphics[width=0.8\textwidth]{Figures/Ch08_BlackHoles/plot_horizon.png}
    \caption{The Supersonic Sink profile showing velocity $v(r)$ crossing $c$ at the Event Horizon.}
    \label{fig:blackhole_profile}
\end{figure}
