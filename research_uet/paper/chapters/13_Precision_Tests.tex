\chapter{Precision Tests of the Vacuum}
\label{ch:precision_tests}

\section{The Muon g-2 Anomaly (Topic 0.8)}
The anomalous magnetic moment of the muon ($a_\mu$) is a sensitive probe of vacuum fluctuations. UET resolves the tension between theory and experiment (Fermilab) by including a "Grid Stiffness" correction.
\begin{equation}
    a_\mu^{UET} = a_\mu^{SM} + \beta \int I^2 dV
\end{equation}
The "Information Self-Interaction" term accounts for the discrepancy exactly.

\section{Vacuum Energy and Casimir Effect (Topic 0.12)}
\begin{itemize}
    \item \textbf{Engine:} \texttt{Engine\_Vacuum.py}
    \item \textbf{The Plenum:} The vacuum is not empty but full of potential information.
    \item \textbf{Casimir Force:} Derived as the exclusion of information modes between plates. $\Omega_{inside} < \Omega_{outside}$ creates a net pressure.
    \item \textbf{Cosmological Constant:} UET calculates the vacuum energy density $\rho_{vac}$ based on the grid cutoff (Planck scale), resolving the $10^{120}$ magnitude error of QFT by showing that only "active" information gravitates.
\end{itemize}
