\chapter{The Nuclear Atom}
\label{ch:nuclear_atom}

\section{Atomic Spectra: Electrons as Standing Waves (Topic 0.20)}
The Hydrogen spectrum is derived without fitting parameters. The electron is modeled as a standing wave on the spherical information grid.
\begin{itemize}
    \item \textbf{Rydberg Constant:} Derived from fundamental geometric parameters.
    \item \textbf{Lamb Shift:} Explained as the interaction of the electron with the background Vacuum Information Field (Topic 0.12).
\end{itemize}

\section{Nuclear Binding: The Geometric Knot (Topic 0.5)}
The Strong Force is not mediated by gluon exchange in UET, but is a result of textual/topological knotting of the nucleon wavefunctions.
\begin{itemize}
    \item \texttt{Engine\_Nuclear\_Binding.py} calculates binding energies using a modified liquid drop model where coefficients are geometric constants.
    \item \textbf{Saturation:} Explained by the capacity limit of the grid nodes within the nucleus.
\end{itemize}

\section{Heavy Nuclei and Stability (Topic 0.16)}
\begin{itemize}
    \item \textbf{Magic Numbers:} The stability peaks at 2, 8, 20, 28, 50, 82, 126 are derived from the optimal packing of spheres in the 3D lattice (Kepler Conjecture analog).
    \item \textbf{Island of Stability:} UET predicts a stable region for superheavy elements based on higher-order geometric symmetries.
\end{itemize}
